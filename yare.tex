	\documentclass{book}
% pre'ambulo

%\newcommand\teacher[1]{\begin{flushright}\small\textit{#1}\end{flushright}}

\usepackage[spanish,activeacute]{babel}
\usepackage{anysize}

\title{Lenguaje de Programaci\'on Yare}
\author{Leonardo Guti\'errez Ram\'irez\\Instituto Tecnológico de Chihuahua II}
\date{17 de Junio del 2012}
\topmargin 0in
\textheight 3.5in
\oddsidemargin 0in
\parindent 1em
\pagenumbering{arabic}
\setcounter{page}{1}
%\marginsize{left}{right}{top}{bottom}:
\marginsize{1cm}{1cm}{1cm}{1cm}
%\renewenvironment{titlepage}{}{}

\begin{document}
% cuerpo del documento

\maketitle

\subsection{?`Por qu\'e otro lenguaje de programaci\'on?}

Mi gusto por la programaci\'on comenz\'o con la lectura de algunos art\'iculos que curioseando en Internet me llegu\'e a topar.\\
\\Dichos art\'iculos trataban sobre un "lenguaje" que interpretaba archivos en donde se pon\'ia una serie de c\'odigos 'extra\~nos'  y la\\
m\'aquina interpertaba, este lenguaje es {\bf Batch}. Un lenguaje arcaico bastante limitado (con tan solo decir que no tiene soporte para 
n\'umeros decimales) pero en el que era o es divertido para los principiantes realizar peque\~nos c\'odigos y tener resultados sobre el Sistema Operativo
relativamente r\'apidos. Programando en Batch me divert\'ia y esto cultiv\'o en m\'i un amor hac\'ia la programaci\'on. Sin embargo, siempre
me sent\'i limitado por dicho lenguaje, por eso he decidido intentar crear mi propio lenguaje de programaci\'on.\\\\


No espero competir con los dem\'as lenguajes de programaci\'on, ser\'ia una tonter\'ia, lo que s\'i espero es que los que est\'en intentando
crear algo parecido a un lenguaje, encuentren un camino m\'as f\'acil con el que yo me top\'e. Es por eso que el c\'odigo de {\bf Yare} est\'a
disponible para modificarlo y claro si ustedes gustan contribuir, pues !`adelante!\\

Programar un lenguaje es una tarea bastante dif\'icil(por lo menos as\'i me ha parecido) y agotadora, requiere de an\'alisis y sobre todo de 
mucho tiempo. Sin embargo, pienso que es una de las tareas que como programador te deja m\'as experiencias.

\subsection{?`Qu\'e es Yare?}

{\bf Yare} es un lenguaje de programaci\'on de alto nivel enfocado a personas que se inician en el hermoso arte de la programaci\'on de 
Software.

Yare es bastante limitado(en parte por mis limitaciones como programador), pero crecer\'a con el tiempo, si es que el tiempo me lo permite.


% \teacher{Prof.~Knuth}
\subsection{Sobre los errores . . .}

No dudo que el c\'odigo contenga errores y d\'e resultados inesperados, por lo que agradecer\'ia que me lo hicieran saber por medio de mi
correo electr\'onico.

{\bf leorocko13@hotmail.com}

%\titlepage{\LARGE \textbf{Caracter\'isticas}}
%\linebreak
\newpage
{\bf \LARGE Caracter\'isticas}
\\\\{\bf \large Tipos de datos}\\\\

Por ahora no hay varios tipos de datos, el \'unico tipo de dato soportado es {\bf double} $(8\ bytes )$ \linebreak
\linebreak Aunque se asignen variables con un tipo entero:\\
$x = -4536$\\Esta se asigna internamente como
$x = -4536{.}0000$ .\\

{\bf \large Variables y su declaraci\'on}\\\\

{\bf Yare} posee por defecto $26$ variables, estas son las variables $a\ ...\ z$, las cuales tambi\'en son de tipo $double$.\\

Para declarar variables basta con escribir un identificador entre los caracteres '$:$', de esta manera:\\
$:edad:\ \ =\ 13343;$\\
$:posicion:\ \ =\ 0{.}003$;\\
$etc$...\\\\
Hay varias maneras de asignar un valor a las variables, estas pueden ser por medio de: los smileys, con un valor aleatorio o por medio del valor devuelto
por una funci\'on.

\begin{itemize}
\item {\bf Smileys.}
\item {\bf Aleatorios.}
\item {\bf Valor devuelto por una funcin.}
\end{itemize}

\textbf{Smileys}

% =====================================================================

Las inicializaciones de variables por smileys se dan de la siguiente manera:\\\\
$:id:|variable\ = \:)$

\end{document}

